\chapter{Analyse de flot de contrôle}

%%%%%%%%%%%%%%%%%%%%%%%%%%%%%%%%%%%%%%%%%%%%%%%%%%%%%%%%%%%%%%%%%%%%%%%%%%%%%%%
\section{Introduction}

L'analyse de flot de données est une partie importante des
compilateurs optimisant. Elle permet d'obtenir de l'information sur
les endroits du programme vers lesquels une certaine valeur peut se
propager. Cette information peut être utilisée par la suite comme
point de départ de toute une panoplie de technique d'optimisation.

Pour pouvoir effectuer l'analyse de flot de données, on doit avoir
sous la main le graphe de flot de contrôle. Dans les langages
statiques comme C, le graphe de flot de contrôle est lui aussi
statique. Par contre, dans les langages dynamiques comme Scheme, il
est plus difficile à obtenir à cause du fait que les fonctions sont
des objets de premier ordre qui peuvent être passés en arguments,
renvoyés en valeur de retour et entreposés dans des structures de
données.

Les fonctions étant des données, il est nécessaire de fait l'analyse de
flot de données pour obtenir le graphe de contrôle. Mais l'analyse de
flot de données nécessite les graphe de contrôle. On a donc un
problème circulaire qui doit être résolu par une technique itérative,
l'analyse de flot de contrôle.

%\clearpage

\section{Analyse de flot de contrôle}

L'analyse de flot de contrôle sert à trouver, pour tout site d'appel
contenu dans le programme, quels fonctions peuvent y être
appelées. Contrairement au langage C ou l'arbre de syntaxe communique
cette information très simplement, dans le langage Scheme n'importe
quelle fonction peut potentiellement se retrouver à un site d'appel
particulier. Pour trouver ces information, nous utiliserons une
variante de la technique 0-cfa décrite dans \cite{shivers88, shivers91}.

\subsection{Ensembles d'approximations}

Notre analyse sur des ensembles d'approximation pouvant contenir les
types de valeurs possibles dans un programme Scheme. Nous avons
volontairement limité la précision de nos approximation dans le but
d'effectuer l'analyse de manière rigoureuse et pouvoir améliorer la
précision par la suite.

Nous avons comme types d'approximations les booléens vrai et faux, les
nombres entiers négatifs, nul et positifs, les nombres flottants
négatifs, nul et positifs, ainsi qu'un type représentant tous les
nombres. Nous avons aussi inclus des types vecteur, fonction,
primitives et le type xvalue représentant toutes valeurs pouvant
provenir de l'extérieur du programme.

Enfin, il y a aussi un type paire mais qui est plus complexe que les
autres types car nous effectuons l'analyse de flot de contrôle et de
données à l'intérieur des paire.  En effet, à chaque lieu d'allocation
cons correspond une approximation paire qui possèdent deux ensemble
d'approximation : car et cdr. Ces ensembles contiennent toutes les
approximation de tous les types ayant pu potentiellement être
entreposés dans une paire allouée à ce lieu d'allocation lors de la
création ou bien de manière subséquente par le biais des primitives
set-car! et set-cdr!. Il est certain que cette estimation est assez
grossière mais elle permet quand même d'obtenir des informations
intéressantes et d'éviter d'obtenir une approximation xvalue beaucoup
trop conservatrice lors de l'appel à un accesseur de paire.

\subsection{Algorithme}

L'algorithme est itératif à point fixe. Il effectue son analyse tant
que des modifications sont ajoutées aux résultats. C'est pourquoi il
ne doit jamais rien enlever aux ensembles d'approximation et que
toutes les valeurs d'approximation possible doivent faire partie d'un
ensemble fini. Cela garantie que l'algorithme devra terminer un jour.

À chaque itération de l'algorithme, on analyse les expressions racines
du programme et on annote les variables ainsi que les sites d'appel
avec les bons ensemble d'approximations. On entre seulement dans le
corps d'une fonction si elle fait partie d'un ensemble d'approximation
qui se retrouve en position d'appel. On marque les fonctions dont on a
inspecté le corps pour ne jamais le faire plus d'une fois dans la même
itération. Toutefois, on annote toujours ses variables paramètres à
chaque fois qu'on la rencontre à un site d'appel.

On conserve un ensemble ESCAPED qui contient les fonction et les
paires ayant échappé au programme. On initialise cet ensemble avec les
ensembles d'estimation des variables globales. Par la suite, a chaque
itération, on le met a jour en 'simulant' un appel a chacune des
fonctions qu'il contient avec comme valeurs de paramètres ce même
ensemble escape. On l'annote avec les résultats de ces appels. De
plus, on l'annote avec les ensembles d'approximation des champs car et
cdr de toutes les paires qu'il contient. Ainsi, notre analyse de
contrôle de flot tient compte de ce qui peut se passer à l'extérieur
du programme, dans la mesure du possible.

%\clearpage

\section{Évaluation Partielle}

\subsection{Analyse des paires}

Les estimations des paires sont très utiles pour améliorer l'analyse
du flot de contrôle mais elles remplissent aussi un autres rôle. Le
but initial était de pouvoir déterminer quelles paires sont constantes
dans tout le programme pour pouvoir effectuer du constant folding sur
ces dernières. Nous avons donc ajouté, en plus des estimations, une
gestion des primitives de mutation des paires. Comme une estimation
paire est relié à un lieu d'appel à une primitive d'allocation, nous
pouvons annoter ce lieu d'appel a chaque fois que son approximation
correspondante est passée à un mutateur set-car! ou set-cdr!. Nous
devons aussi le faire pour la primitive eq?  pour traiter les cas tels
que (eq? (list 1) (list 1)). En annotant ainsi, nous pouvons, dans le
système de constant folding de gambit, vérifier si une approximation
correspondant au noeud qu'on tente d'évaluer à la compilation, a été
passée à une de ces primitives. Si ce n'est pas le cas, nous avons
l'information nécessaire pour affirmer que cette paire peut être
traitée comme une constante et donc faire l'objet d'évaluation
partielle.

\subsection{Élimination des tests}

Le constant folding permet d'éliminer à priori plusieurs tests de
manière naïve. Malheureusement, il a ses limitations en l'absence
d'une analyse plus poussée. En effet, pour pouvoir réduire un tel
test, son argument doit être une constante.  Notre analyse permet
d'ajouter d'autre cas ou ces tests pourront être éliminés. En effet,
comme nous déterminons des ensemble d'approximation et que ces
ensemble contiennent des approximation de types, on peut grâce à ces
approximations éliminer certains tests en s'assurant que tous les
types présents dans l'ensemble donnerait le même résultat.

Pour cela, nous avons ajouté un nouveau type de constant folder qui
filtre sur les approximation au lieu des constantes.  Nous devons
toutefois nous limiter aux appels de primitive fait avec des variables
comme paramètre car nous risquerions de faire disparaître certain
effets de bord en réduisant un sous-arbre et ainsi modifier la
sémantique du programme.

%\clearpage

\section{Exemples}

Voici quelques exemples qui illustre les optimisations possibles grâces aux
nouvelles information que notre analyse fourni à la béta-réduction.

\subsection{\og Constant folding \fg de tests}

Cet exemple vise a démontrer la capacité du système d'utiliser les
ensemble d'approximation afin de réduire certaines primitives de
test. Dans l'exemple de la figure \ref{cfa-ex1-source}, la beta
réduction telle qu'elle est implémentée dans Gambit-C ne peut déduire
que la valeur du test de paire sera toujours vrai.  Avec notre
analyse, nous avons pu introduire un nouveau réducteur qui vérifie
l'ensemble d'approximation de la variable x et réalise que toutes les
approximations sont des paires. il peut donc réduire le test a true.

  \begin{figure}[htbp!]
    \begin{lstlisting}
(let ((x (if j (cons 1 2) (cons 3 4))))
  (let ((y (##pair? x)))
    y))
    \end{lstlisting}
    \caption{Code source illustrant le \og constant folding \fg de la
      fonction \texttt{pair?}.}
    \label{cfa-ex1-source}
  \end{figure}

On peut voir comment la réduction a été effectuée dans la figure
\ref{cfa-ex1-out}. On s'aperçoit tout de suite que l'élimination de
code mort n'a pas enlevé la variable x alors qu'elle n'est pas
utilisée. C'est une des failles de l'approche actuelle qui
bénéficierait de la possibilité de pouvoirs faire plusieurs
itérations.

 \begin{figure}[htbp!]
    \begin{lstlisting}
(let ((x (if j '(1 . 2) '(3 . 4))))
  #t)
    \end{lstlisting}
    \caption{Code obtenu après les phases d'optimisations de Gambit-C à
      partir du code source présenté dans la figure \ref{cfa-ex1-source}.}
    \label{cfa-ex1-out}
  \end{figure}

%\clearpage

\subsection{Optimisations des paires constantes}

Cet exemple tente de démontrer ce que l'analyse des paires constante permet
d'optimiser. Dans l'exemple de la figure \ref{cfa-ex2-source}, on démontre
ce que le \og Constant folding \fg de Gambit-C peut accomplir lorsqu'on lui
donne l'information que la liste lst est constante. Nous savons que cette
liste est constant car notre analyse réalise qu'elle n'est jamais passée à
\texttt{set-car!}, \texttt{set-cdr!} et \texttt{eq?}.

\begin{figure}[htbp!]
  \begin{lstlisting}
(let ((lst (cons 1 (cons 2 (cons 3 '()))))
      (kindof-map (lambda (f lst)
                    (if (<= (length lst) 3)
                        (list (f (car lst))
                              (f (cadr lst))
                              (f (caddr lst)))
                        (map f lst)))))
  (kindof-map (lambda (x) (* x x)) lst))
  \end{lstlisting}
  \caption{Code source illustrant l'optimisation des paires constantes.}
  \label{cfa-ex2-source}
\end{figure}

On peut voir dans la figure \ref{cfa-ex2-out} que le compilateur a totalement
réduit l'appel à \texttt{kindof-map}. En effet, sachant que la liste était
constante, il a pu réduire le \texttt{let} au corp de \texttt{kindof-map} et
réduire le \texttt{if} à sa branche vrai sachant quelle était la taille de la liste.
Il reste encore certaines autres choses qui pourrait être optimisées dans cet exemple.
On devrait pouvoir obtenir uniquement une liste constant '(1 4 9).

\begin{figure}[htbp!]
  \begin{lstlisting}
(let ((f (lambda (x) (* x x))))
  ('#<procedure #8 ##list> (f 1) (f 2) (f 3)))
  \end{lstlisting}
  \caption{Code généré à partir du code source fourni dans la figure
    \ref{cfa-ex2-source} par le compilateur Gambit-C en utilisant les
    résultats de notre analyse \textit{0-cfa} et notre détection des
    paires constantes.}
  \label{cfa-ex2-out}
\end{figure}

%\clearpage

\section{Travail futur}

Un système d'optimisation n'est jamais terminé et les résultats que
nous obtenons grâce a cette analyse ne sont que le début de ce qui
pourrait être fait. L'analyse de flot de contrôle elle même est
complète.  C'est surtout l'analyse de flot de données qui pourrait
bénéficier d'améliorations

\subsection{Multiples itérations}

Une première limitation de notre système, qu'on découvre lorsqu'on
observer certains exemples est que d'autres informations pourraient être
obtenue si on effectuait de nouveau l'analyse sur du code déjà optimisé
Avec ces informations, il serait possible de procéder a d'autres optimisations
et réductions du code. Bien entendu, cela ne se fait pas sans coût et il
faut savoir ce qu'on est prêt a payer en temps de compilation. Malgré tout
il devrait être optionellement possible de pouvoir effectuer l'analyse
un certain nombre de fois arbitraire ou même de tenter d'atteindre un point
fixe au niveau du code.

\subsection{Analyse de flot sur les vecteurs}

Notre système effectue l'analyse de flot de contrôle a l'intérieur des paires
et effectue aussi l'analyse de flot de données sur ces dernières. Il serait
possible et même bénéfique d'effectuer les mêmes analyses sur les vecteurs.
Cela nous permettrait de suivre les fonctions à l'intérieur des vecteurs
et aussi de détecter les vecteurs constants qui pourrait être réduit à
la compilation. Cette analyse ne serait pas très complexe a implémenter.

\subsection{Filtrage de types}

Une amélioration très intéressantes serait d'utiliser le
filtrage de types tel que décrit dans \cite{shivers88}.  Cette
technique nous permettrait d'augmenter la précision de nos ensembles
d'approximation dans les branches d'une conditionelle en fonction du
test de typage qu'elle effectue. On peut aussi utiliser les appels aux
primitives de la même façon car celles-ci contiennent des tests de
types cachés qui nous permettre aussi de préciser notre ensemble si
elle retourne. Une implémentation de cette technique est celle décrite
par \cite{boucher2000} et est plus simple que l'approche générale que
Shivers adopte dans son article.

\subsection{Réduction d'intervalles}

Une autre technique très utile serait le 'interval narrowing' aussi
implémentée par \cite{boucher2000}.  Il s'agit d'utiliser les
primitives booléennes pour augmenter la précision d'une approximation
intervalle dans une branche. Ces deux optimisations combinées pourrait
être très utiles pour éliminer une grande quantité de tests de
typages sur les fixnums et les flonums et ainsi grandement accélérer
le code.

\subsection{Système de module}

Finalement, notre système ainsi que le compilateur gambit-c
bénéficierait énormément de l'ajout d'un système de module permettant
l'encapsulation des fonctions et l'export explicite. Cela nous
permettrait d'augmenter le nombre d'optimisation possible pour
certaine fonction interne du au fait qu'elle n'echapperait pas au
module et donc qu'elle ne pourrait pas etre appelee avec des valeurs
externes. Cela augmenterait sensiblement la précision de notre
analyse.

%\clearpage

\section{Conclusion}

Nous avons réussi, lors de ce travail, à implémenter avec succès une
analyse de flot de contrôle dans le système Gambit-C. Nous avons aussi
implémenter une analyse de flot de données qui pourra être améliorée
en augmentant la précision des estimations qu'elle produit. Notre
analyse suis entre autre le flot des paires ce qui permet de faire
certaines évaluations partielles sur les paires constantes. Il a aussi
permis de faire des optimisations en permettant d'éliminer certains
tests redondant.

Nous croyons que notre implémentation est un bon point de départ pour
permettre l'implémentation de techniques plus poussées d'optimisation
dans le système Gambit-C. Éventuellement, la beta réduction du système
actuel pourrait utiliser l'analyse de flot de données de notre
implémentation au lieu d'avoir deux analyses différentes. Nous
souhaitons de tout coeur que notre travail soit intégré au système
Gambit-C.
