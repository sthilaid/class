\documentclass[letterpaper,12pt]{book}
\usepackage[utf8]{inputenc}
\usepackage{graphicx, subfigure}
\usepackage{longtable}
\usepackage{amsmath,amsfonts}
\usepackage{array}
\usepackage{xspace}
\usepackage{color, fancyvrb, relsize}
\usepackage{url, hyperref}
\usepackage{algorithm, algorithmic, listings}
\usepackage[francais]{babel}
\selectlanguage{francais}

%ref: http://tex.dante.jp/typo/index.php?Listings#l8795a28

\usepackage{color,graphicx}
%\usepackage[margin=2.5cm]{geometry}
%\usepackage{type1cm}
\usepackage{listings}

\lstdefinelanguage{Scheme}{%
   keywords={begin,call-with-current-continuation,call/cc,%
      call-with-input-file,call-with-output-file,case,cond,do,%
      delay,else,force,for-each,if,lambda,let,let-syntax,let*,%
      letrec,letrec-syntax,and,or,not,map,syntax,syntax-rules,%
      with-mode,with-input-from-file,with-input-fromport,%
      with-output-to-file,with-output-to-port,define,%
			set!, eq?, equal?, eqv?, symbol?, string?, string=?,
			append, string-append, vector-append, string, vector,
			symbol, symbol->string, string->symbol, list, list-ref,
			list-set!, list-tail, set-car!, set-cdr!, car, cdr,
			vector-ref, vector-set!, define-class, define-method, 
      define-generic
   },%
   sensitive,%
   alsodigit=-,%
   morecomment=[l]{;},% 
   morecomment=[s]{\#|}{|\#},% 
   morestring=[b]{"}% 
  }[keywords,comments,strings]%"

\lstdefinelanguage[MIT]{Scheme}{%
   morekeywords=[1]{begin,call-with-current-continuation,call/cc,%
      call-with-input-file,call-with-output-file,case,cond,do,%
      delay,else,force,for-each,if,lambda,let,let-syntax,let*,%
      letrec,letrec-syntax,and,or,not,map,syntax,syntax-rules,%
      with-mode,with-input-from-file,with-input-fromport,%
      with-output-to-file,with-output-to-port,define%
   },%
   morekeywords=[2]{fluid-let,%
     in-package,local-declare,macro,make-environment,named-lambda,%
     using-syntax,with-input-from-string,with-output-to-string,%
     with-values,syntax-table-define,list-transform-positive,%
     list-transform-negative,list-search-positive,list-search-negative,%
     access-components,assignment-components,combination-components,%
     comment-components,conditional-components,disjunction-components,%
     declaration-components,definition-components,delay-components,%
     in-package-components,lambda-components,lambda-components*,%
     lambda-components**,open-block-components,pathname-components,%
     procedure-components,sequence-components,unassigned?-components,%
     unbound?-components,variable-components,%
     \#t,\#f,\#e,\#i,\#b,\#o,\#d,\#l,\#s,\#x,\#!\#*,\#@,\#=,\#\#,%
     access,cons-stream,declare,default-object?,define-integrable,%
     define-structure,define-syntax,er-macro-transformer,%
     non-hygienic-macro-transformer,quasiquote,quote,%
     rsc-macro-transformer,sc-macro-transformer,set!,%
     eq?,eqv?,equal?,%
     gcd,modulo,imag-part,numerator,inexact->exact,quotient,abs,%
     lcm,rationalize,angle,magnitude,real-part,ceiling,make-polar,%
     remainder,denominator,make-rectangular,round,expt,max,truncate,%
     floor,min%
  }%
   sensitive=true,%
   alsodigit=-,%
   morecomment=[l]{;},% 
   morecomment=[s]{\#|}{|\#},% 
   morestring=[b]{"}% "
}

\lstset{%
 frame=single,%
 keywordstyle={\ttfamily \color[rgb]{.1, .7, .7}},%
 stringstyle={\ttfamily \color[rgb]{0,.9,0}},%
 commentstyle={\ttfamily \color[rgb]{.9,0,0}},%
 identifierstyle={\color[rgb]{0,0,0}},%
 basicstyle={\ttfamily \color[rgb]{0,0,0}},%
 breaklines=true,%
 columns=[l]{fullflexible},%
 numbers=left,%
 %stepnumber=5,%
 numberstyle={\scriptsize},%
 numbersep=1em,%
 language={Scheme},%
}
\newcommand*\schemesrc[1]{\lstinputlisting[caption={question~#1}]{#1.scm}}


\lstloadlanguages{Scheme}
\lstset{language=Scheme}
\textheight=600pt

% commande qui fait en sorte que seul les Section sont numerottees
%% \setcounter{secnumdepth}{1}
% \setcounter{page}{0}
\newcommand{\forcenewline}[0]{\begin{center}\end{center}}

\begin{document}

\title{IFT6232 - Étape 3: Optimisations d'évaluations partielles et
  système d'objets polymorphes pour Gambit-C}

\author{par Jérémie Lasalle Ratelle et David St-Hilaire}
\maketitle

%% \clearpage

\tableofcontents

%% \clearpage


\chapter{Analyse de flot de contrôle}

%%%%%%%%%%%%%%%%%%%%%%%%%%%%%%%%%%%%%%%%%%%%%%%%%%%%%%%%%%%%%%%%%%%%%%%%%%%%%%%
\section{Introduction}

L'analyse de flot de données est une partie importante des
compilateurs optimisant. Elle permet d'obtenir de l'information sur
les endroits du programme vers lesquels une certaine valeur peut se
propager. Cette information peut être utilisée par la suite comme
point de départ de toute une panoplie de technique d'optimisation.

Pour pouvoir effectuer l'analyse de flot de données, on doit avoir
sous la main le graphe de flot de contrôle. Dans les langages
statiques comme C, le graphe de flot de contrôle est lui aussi
statique. Par contre, dans les langages dynamiques comme Scheme, il
est plus difficile à obtenir à cause du fait que les fonctions sont
des objets de premier ordre qui peuvent être passés en arguments,
renvoyés en valeur de retour et entreposés dans des structures de
données.

Les fonctions étant des données, il est nécessaire de fait l'analyse de
flot de données pour obtenir le graphe de contrôle. Mais l'analyse de
flot de données nécessite les graphe de contrôle. On a donc un
problème circulaire qui doit être résolu par une technique itérative,
l'analyse de flot de contrôle.

%\clearpage

\section{Analyse de flot de contrôle}

L'analyse de flot de contrôle sert à trouver, pour tout site d'appel
contenu dans le programme, quels fonctions peuvent y être
appelées. Contrairement au langage C ou l'arbre de syntaxe communique
cette information très simplement, dans le langage Scheme n'importe
quelle fonction peut potentiellement se retrouver à un site d'appel
particulier. Pour trouver ces information, nous utiliserons une
variante de la technique 0-cfa décrite dans \cite{shivers88, shivers91}.

\subsection{Ensembles d'approximations}

Notre analyse sur des ensembles d'approximation pouvant contenir les
types de valeurs possibles dans un programme Scheme. Nous avons
volontairement limité la précision de nos approximation dans le but
d'effectuer l'analyse de manière rigoureuse et pouvoir améliorer la
précision par la suite.

Nous avons comme types d'approximations les booléens vrai et faux, les
nombres entiers négatifs, nul et positifs, les nombres flottants
négatifs, nul et positifs, ainsi qu'un type représentant tous les
nombres. Nous avons aussi inclus des types vecteur, fonction,
primitives et le type xvalue représentant toutes valeurs pouvant
provenir de l'extérieur du programme.

Enfin, il y a aussi un type paire mais qui est plus complexe que les
autres types car nous effectuons l'analyse de flot de contrôle et de
données à l'intérieur des paire.  En effet, à chaque lieu d'allocation
cons correspond une approximation paire qui possèdent deux ensemble
d'approximation : car et cdr. Ces ensembles contiennent toutes les
approximation de tous les types ayant pu potentiellement être
entreposés dans une paire allouée à ce lieu d'allocation lors de la
création ou bien de manière subséquente par le biais des primitives
set-car! et set-cdr!. Il est certain que cette estimation est assez
grossière mais elle permet quand même d'obtenir des informations
intéressantes et d'éviter d'obtenir une approximation xvalue beaucoup
trop conservatrice lors de l'appel à un accesseur de paire.

\subsection{Algorithme}

L'algorithme est itératif à point fixe. Il effectue son analyse tant
que des modifications sont ajoutées aux résultats. C'est pourquoi il
ne doit jamais rien enlever aux ensembles d'approximation et que
toutes les valeurs d'approximation possible doivent faire partie d'un
ensemble fini. Cela garantie que l'algorithme devra terminer un jour.

À chaque itération de l'algorithme, on analyse les expressions racines
du programme et on annote les variables ainsi que les sites d'appel
avec les bons ensemble d'approximations. On entre seulement dans le
corps d'une fonction si elle fait partie d'un ensemble d'approximation
qui se retrouve en position d'appel. On marque les fonctions dont on a
inspecté le corps pour ne jamais le faire plus d'une fois dans la même
itération. Toutefois, on annote toujours ses variables paramètres à
chaque fois qu'on la rencontre à un site d'appel.

On conserve un ensemble ESCAPED qui contient les fonction et les
paires ayant échappé au programme. On initialise cet ensemble avec les
ensembles d'estimation des variables globales. Par la suite, a chaque
itération, on le met a jour en 'simulant' un appel a chacune des
fonctions qu'il contient avec comme valeurs de paramètres ce même
ensemble escape. On l'annote avec les résultats de ces appels. De
plus, on l'annote avec les ensembles d'approximation des champs car et
cdr de toutes les paires qu'il contient. Ainsi, notre analyse de
contrôle de flot tient compte de ce qui peut se passer à l'extérieur
du programme, dans la mesure du possible.

%\clearpage

\section{Évaluation Partielle}

\subsection{Analyse des paires}

Les estimations des paires sont très utiles pour améliorer l'analyse
du flot de contrôle mais elles remplissent aussi un autres rôle. Le
but initial était de pouvoir déterminer quelles paires sont constantes
dans tout le programme pour pouvoir effectuer du constant folding sur
ces dernières. Nous avons donc ajouté, en plus des estimations, une
gestion des primitives de mutation des paires. Comme une estimation
paire est relié à un lieu d'appel à une primitive d'allocation, nous
pouvons annoter ce lieu d'appel a chaque fois que son approximation
correspondante est passée à un mutateur set-car! ou set-cdr!. Nous
devons aussi le faire pour la primitive eq?  pour traiter les cas tels
que (eq? (list 1) (list 1)). En annotant ainsi, nous pouvons, dans le
système de constant folding de gambit, vérifier si une approximation
correspondant au noeud qu'on tente d'évaluer à la compilation, a été
passée à une de ces primitives. Si ce n'est pas le cas, nous avons
l'information nécessaire pour affirmer que cette paire peut être
traitée comme une constante et donc faire l'objet d'évaluation
partielle.

\subsection{Élimination des tests}

Le constant folding permet d'éliminer à priori plusieurs tests de
manière naïve. Malheureusement, il a ses limitations en l'absence
d'une analyse plus poussée. En effet, pour pouvoir réduire un tel
test, son argument doit être une constante.  Notre analyse permet
d'ajouter d'autre cas ou ces tests pourront être éliminés. En effet,
comme nous déterminons des ensemble d'approximation et que ces
ensemble contiennent des approximation de types, on peut grâce à ces
approximations éliminer certains tests en s'assurant que tous les
types présents dans l'ensemble donnerait le même résultat.

Pour cela, nous avons ajouté un nouveau type de constant folder qui
filtre sur les approximation au lieu des constantes.  Nous devons
toutefois nous limiter aux appels de primitive fait avec des variables
comme paramètre car nous risquerions de faire disparaître certain
effets de bord en réduisant un sous-arbre et ainsi modifier la
sémantique du programme.

%\clearpage

\section{Exemples}

Voici quelques exemples qui illustre les optimisations possibles grâces aux
nouvelles information que notre analyse fourni à la béta-réduction.

\subsection{\og Constant folding \fg de tests}

Cet exemple vise a démontrer la capacité du système d'utiliser les
ensemble d'approximation afin de réduire certaines primitives de
test. Dans l'exemple de la figure \ref{cfa-ex1-source}, la beta
réduction telle qu'elle est implémentée dans Gambit-C ne peut déduire
que la valeur du test de paire sera toujours vrai.  Avec notre
analyse, nous avons pu introduire un nouveau réducteur qui vérifie
l'ensemble d'approximation de la variable x et réalise que toutes les
approximations sont des paires. il peut donc réduire le test a true.

  \begin{figure}[htbp!]
    \begin{lstlisting}
(let ((x (if j (cons 1 2) (cons 3 4))))
  (let ((y (##pair? x)))
    y))
    \end{lstlisting}
    \caption{Code source illustrant le \og constant folding \fg de la
      fonction \texttt{pair?}.}
    \label{cfa-ex1-source}
  \end{figure}

On peut voir comment la réduction a été effectuée dans la figure
\ref{cfa-ex1-out}. On s'aperçoit tout de suite que l'élimination de
code mort n'a pas enlevé la variable x alors qu'elle n'est pas
utilisée. C'est une des failles de l'approche actuelle qui
bénéficierait de la possibilité de pouvoirs faire plusieurs
itérations.

 \begin{figure}[htbp!]
    \begin{lstlisting}
(let ((x (if j '(1 . 2) '(3 . 4))))
  #t)
    \end{lstlisting}
    \caption{Code obtenu après les phases d'optimisations de Gambit-C à
      partir du code source présenté dans la figure \ref{cfa-ex1-source}.}
    \label{cfa-ex1-out}
  \end{figure}

%\clearpage

\subsection{Optimisations des paires constantes}

Cet exemple tente de démontrer ce que l'analyse des paires constante permet
d'optimiser. Dans l'exemple de la figure \ref{cfa-ex2-source}, on démontre
ce que le \og Constant folding \fg de Gambit-C peut accomplir lorsqu'on lui
donne l'information que la liste lst est constante. Nous savons que cette
liste est constant car notre analyse réalise qu'elle n'est jamais passée à
\texttt{set-car!}, \texttt{set-cdr!} et \texttt{eq?}.

\begin{figure}[htbp!]
  \begin{lstlisting}
(let ((lst (cons 1 (cons 2 (cons 3 '()))))
      (kindof-map (lambda (f lst)
                    (if (<= (length lst) 3)
                        (list (f (car lst))
                              (f (cadr lst))
                              (f (caddr lst)))
                        (map f lst)))))
  (kindof-map (lambda (x) (* x x)) lst))
  \end{lstlisting}
  \caption{Code source illustrant l'optimisation des paires constantes.}
  \label{cfa-ex2-source}
\end{figure}

On peut voir dans la figure \ref{cfa-ex2-out} que le compilateur a totalement
réduit l'appel à \texttt{kindof-map}. En effet, sachant que la liste était
constante, il a pu réduire le \texttt{let} au corp de \texttt{kindof-map} et
réduire le \texttt{if} à sa branche vrai sachant quelle était la taille de la liste.
Il reste encore certaines autres choses qui pourrait être optimisées dans cet exemple.
On devrait pouvoir obtenir uniquement une liste constant '(1 4 9).

\begin{figure}[htbp!]
  \begin{lstlisting}
(let ((f (lambda (x) (* x x))))
  ('#<procedure #8 ##list> (f 1) (f 2) (f 3)))
  \end{lstlisting}
  \caption{Code généré à partir du code source fourni dans la figure
    \ref{cfa-ex2-source} par le compilateur Gambit-C en utilisant les
    résultats de notre analyse \textit{0-cfa} et notre détection des
    paires constantes.}
  \label{cfa-ex2-out}
\end{figure}

%\clearpage

\section{Travail futur}

Un système d'optimisation n'est jamais terminé et les résultats que
nous obtenons grâce a cette analyse ne sont que le début de ce qui
pourrait être fait. L'analyse de flot de contrôle elle même est
complète.  C'est surtout l'analyse de flot de données qui pourrait
bénéficier d'améliorations

\subsection{Multiples itérations}

Une première limitation de notre système, qu'on découvre lorsqu'on
observer certains exemples est que d'autres informations pourraient être
obtenue si on effectuait de nouveau l'analyse sur du code déjà optimisé
Avec ces informations, il serait possible de procéder a d'autres optimisations
et réductions du code. Bien entendu, cela ne se fait pas sans coût et il
faut savoir ce qu'on est prêt a payer en temps de compilation. Malgré tout
il devrait être optionellement possible de pouvoir effectuer l'analyse
un certain nombre de fois arbitraire ou même de tenter d'atteindre un point
fixe au niveau du code.

\subsection{Analyse de flot sur les vecteurs}

Notre système effectue l'analyse de flot de contrôle a l'intérieur des paires
et effectue aussi l'analyse de flot de données sur ces dernières. Il serait
possible et même bénéfique d'effectuer les mêmes analyses sur les vecteurs.
Cela nous permettrait de suivre les fonctions à l'intérieur des vecteurs
et aussi de détecter les vecteurs constants qui pourrait être réduit à
la compilation. Cette analyse ne serait pas très complexe a implémenter.

\subsection{Filtrage de types}

Une amélioration très intéressantes serait d'utiliser le
filtrage de types tel que décrit dans \cite{shivers88}.  Cette
technique nous permettrait d'augmenter la précision de nos ensembles
d'approximation dans les branches d'une conditionelle en fonction du
test de typage qu'elle effectue. On peut aussi utiliser les appels aux
primitives de la même façon car celles-ci contiennent des tests de
types cachés qui nous permettre aussi de préciser notre ensemble si
elle retourne. Une implémentation de cette technique est celle décrite
par \cite{boucher2000} et est plus simple que l'approche générale que
Shivers adopte dans son article.

\subsection{Réduction d'intervalles}

Une autre technique très utile serait le 'interval narrowing' aussi
implémentée par \cite{boucher2000}.  Il s'agit d'utiliser les
primitives booléennes pour augmenter la précision d'une approximation
intervalle dans une branche. Ces deux optimisations combinées pourrait
être très utiles pour éliminer une grande quantité de tests de
typages sur les fixnums et les flonums et ainsi grandement accélérer
le code.

\subsection{Système de module}

Finalement, notre système ainsi que le compilateur gambit-c
bénéficierait énormément de l'ajout d'un système de module permettant
l'encapsulation des fonctions et l'export explicite. Cela nous
permettrait d'augmenter le nombre d'optimisation possible pour
certaine fonction interne du au fait qu'elle n'echapperait pas au
module et donc qu'elle ne pourrait pas etre appelee avec des valeurs
externes. Cela augmenterait sensiblement la précision de notre
analyse.

%\clearpage

\section{Conclusion}

Nous avons réussi, lors de ce travail, à implémenter avec succès une
analyse de flot de contrôle dans le système Gambit-C. Nous avons aussi
implémenter une analyse de flot de données qui pourra être améliorée
en augmentant la précision des estimations qu'elle produit. Notre
analyse suis entre autre le flot des paires ce qui permet de faire
certaines évaluations partielles sur les paires constantes. Il a aussi
permis de faire des optimisations en permettant d'éliminer certains
tests redondant.

Nous croyons que notre implémentation est un bon point de départ pour
permettre l'implémentation de techniques plus poussées d'optimisation
dans le système Gambit-C. Éventuellement, la beta réduction du système
actuel pourrait utiliser l'analyse de flot de données de notre
implémentation au lieu d'avoir deux analyses différentes. Nous
souhaitons de tout coeur que notre travail soit intégré au système
Gambit-C.

%% \clearpage



\chapter{Systèmes d'objets}

%%%%%%%%%%%%%%%%%%%%%%%%%%%%%%%%%%%%%%%%%%%%%%%%%%%%%%%%%%%%%%%%%%%%%%%%%%%%%%%
\section{Introduction}

Les langages de programmations se distinguent par différents concepts
ou caractéristiques qui leurs sont propres. Le langage Scheme est
caractérisé non seulement par son système de macro très évolué et par
les fonctions de premières classes, mais aussi par sa
\emph{simplicité}.

En effet, Scheme est un langage très minimaliste. Il fut conçu de
manière à comprendre tous les outils nécessaire, afin d'en avoir aussi
peu que possible, contrairement à Lisp où l'approche inverse est
utilisée. Ainsi, pour programmer en Scheme des programmes non
triviaux, plusieurs librairies auxiliaires doivent être généralement
écrites. Par contre, puisque le langage est très expressifs, ces
librairies peuvent être écrites rapidement.

Un module intéressant à ajouter au dessus de Scheme est un système
orienté objet. En effet, un tel système permet la création et
l'utilisation d'objets de manières très dynamique et expressive par,
entre autre, l'utilisation du polymorphisme et de l'héritage.

Nous proposons donc pour la troisième étape de notre projet d'écrire
un système d'objets pouvant s'ajouter à Scheme pour d'une part,
obtenir un module utile pour tout programmeur Scheme et, d'autre part,
démontrer qu'il est possible d'écrire rapidement un système qui peut
sembler très complexe à première vu en utilisant la puissance
expressive de Scheme.

%\clearpage


%%%%%%%%%%%%%%%%%%%%%%%%%%%%%%%%%%%%%%%%%%%%%%%%%%%%%%%%%%%%%%%%%%%%%%%%%%%%%%%

  \section{Besoins et Contraintes}
    Le système de programmation orienté objet conçu respecte plusieurs
    contraintes dans le but de répondre à des besoins
    correspondant. Ces contraintes sont donc indiquées dans cette
    section en ordre d'importance.

    \subsubsection{Rapidité d'accès aux champs}
      Cette contrainte est la plus importante pour notre implantation
      du système objet. En effet, les accès aux champs des objets sont
      des opérations extrêmement fréquentes et donc, afin d'avoir un
      système utilisable à l'intérieur d'applications concrètes, ces
      accès doivent se faire le plus rapidement possible afin de
      diminuer l'impacte du coût de l'utilisation d'un système
      d'objets.

    \subsubsection{Rapidité de choix d'instances de fonction générique}
      De manière similaire à la contrainte reliée à l'accès aux
      champs, notre système objets doit permettre d'obtenir rapidement
      l'instance correspondant à un appel de fonction générique. Cette
      contrainte impose donc aux fonctions générique de ne pas évaluer
      dynamiquement quelle serait la meilleure instance d'une fonction
      générique pour les types des objets passés en paramètres.

    \subsubsection{Modularité}
      Le système doit être facile à intégrer à une application écrite
      en Scheme. Donc elle devrait pouvoir être incluse comme toute
      librairie et ne pas être intégrée directement dans une
      implantation de Scheme, pour faciliter l'utilisation.

    \subsubsection{Interface à la CLOS}
      Le système d'objet devrait avoir une interface similaire à celle
      fournie par le système objets du langage Common
      Lisp~\cite{COMMONLISP} (Common Lisp Object System, ou encore
      CLOS). Puisque Scheme est une version épurée de Common Lisp et
      puisque CLOS est déjà bien connu et à fait ses preuves en tant
      qu'utilisabilité, il serait donc naturel qu'un système d'objets
      pour Scheme respecte des conventions d'utilisations et des
      concepts similaires à ceux qui sont déjà bien connu chez CLOS,
      afin d'augmenter l'accessibilité du système.
    
    \subsubsection{Développement interactif}
      Il serait intéressant que le système permette de pouvoir faire
      du développement itératif (de manière interactive avec un REPL)
      de manière à ce que le système soit cohérent avec la philosophie
      de développement généralement utilisée en Scheme, \textit{i.e.}
      de régulièrement tester le code écrit de manière interprétée.
      Par contre, le fait de permettre ne devrait pas nuire aux
      performances du système, surtout si on désire que ce système
      soit compilé (et donc exécuté rapidement).

    \subsubsection{Portabilité relative}
      Finalement, il serait pertinent que système soit écrit de
      manière portable afin qu'il puisse être utilisé dans plusieurs
      implantation de Scheme. 


%%%%%%%%%%%%%%%%%%%%%%%%%%%%%%%%%%%%%%%%%%%%%%%%%%%%%%%%%%%%%%%%%%%%%%%%%%%%%%%
  \section{Interface de Programmation (\texttt{API})}
    \label{api}

    Cette section contient un aperçu du système d'objet implanté, et
    décrit brièvement son utilisation.
  
    \subsection{Création de Classes et d'Objets}
      La forme spéciale \texttt{define-class} permet de déclarer une
      nouvelle classe (ou d'en redéfinir une existante). Cette forme
      spéciale possède grossièrement la forme suivante:

      \begin{lstlisting}
(define-method class-name (supers) slots...)
      \end{lstlisting}

      où \texttt{class-name} est le nom de la class définie,
      \texttt{supers} est une liste des super-classes de la classes
      définie et \texttt{slots...} sera des définitions des champs de
      l'objet.

      Les champs peuvent être soient présents dans chaque instances
      (obtenu par \texttt{(slot: nom)} ou commun à toutes les
      instances (obtenu par \texttt{(class-slot: nom)}. De plus, il
      existe des options pour les champs:
      
      \begin{itemize}
      \item[\texttt{read-hook:}] Permet de passer en paramètre
        plusieurs fonctions qui seront appelées avant chaque lecture
        du champs.
      \item[\texttt{write-hook:}] Permet de passer en paramètre
        plusieurs fonctions qui seront appelées avant chaque écriture
        du champs.
      \end{itemize}

      Lorsque ces options sont activées pour des champs d'instances,
      les fonctions doivent avoir une signature comme:
      \texttt{(lambda(obj val) ...)}, sinon si le champs est une
      champs de classe, la signature des fonctions devraient
      ressembler à: \texttt{(lambda(val)...)}.

      Un exemple (figure \ref{ex-defclass}) illustre de manière
      général la définition de classes.

      \begin{figure}[htbp!]
        \begin{lstlisting}
(define-class A () 
  (slot: a (read-hooks: (lambda (o v) (pp v)))))
(define-class B () 
  (slot: b (write-hooks: (lambda (o new-v) (pp new-v)))))
(define-class C (A B) (class-slot: c))
        \end{lstlisting}
        \caption{Exemple de définition de classe}
        \label{ex-defclass}
      \end{figure}

      Une déclaration de classe résulte en la création de plusieurs
      fonctions utilitaires. Par exemple pour une définition comme
      \begin{lstlisting}
(define-class A () (slot: a) (class-slot: b))}
      \end{lstlisting}
      on obtiendrait:

      \begin{itemize}
      \item[\texttt{(make-A a)}:] Fonction de création d'une instance
        de la classe \texttt{A}. Prend en paramètre les valeurs des
        champs d'instances, en commençant par les champs hérités, dans
        l'ordre (récursif) donné des super classes, tel qu'illustré
        dans la figure \ref{ex-make-instance}.
        \begin{figure}[h!]
          \begin{lstlisting}
(define-class A ()   (slot: a))
(define-class B (A)  (slot: b))
(define-class C ()   (slot: c))
(define-class D (A C) (slot: d))
(make-D a b c d)
          \end{lstlisting}
          \caption{Exemple illustrant l'ordre des paramètre dans un
            constructeur d'instances.}
          \label{ex-make-instance}
        \end{figure}
      \item[]

      \item[\texttt{(A? obj)}:] Prédicat qui test si l'objet passé en
        paramètre est une instance ou une sous-classe de \texttt{A}.
      \item[]

      \item[\texttt{(A-a obj)}:] Accès au champs d'instance \texttt{a}
        de l'objet passé en paramètre.
      \item[]

      \item[\texttt{(A-a-set! obj new-a)}:] Modification du champs
        d'instance \texttt{a} de l'objet passé en paramètre.
      \item[]

      \item[\texttt{(A-b)}:] Accès au champs statique \texttt{b}.
      \item[]

      \item[\texttt{(A-b-set! new-b)}:] Modification du champs
        statique \texttt{b}.
      \item[]

      \item[\texttt{(describe obj)}:] Instance de fonction générique
        qui retourne une liste décrivant le contenu de
        l'instance. Utilise pour visualiser le contenu avec
        \texttt{(pp (describe obj))}.
      \end{itemize}

    \subsection{Déclaration et Instanciation de Fonctions Génériques}

      \subsubsection{Déclarations de fonctions génériques}
        La déclaration de fonctions génériques se fait par l'utilisation
        de la forme spéciale suivante:

      \begin{lstlisting}
(define-generic (fname arg1 arg2 ...))
      \end{lstlisting}

        Cette déclaration indique donc la présence d'une nouvelle
        fonction générique avec son arité (et le nom de chacun de ses
        paramètres).

        Par la suite, les instances des fonctions génériques peuvent
        être déclarées par la forme spéciale:

      \begin{lstlisting}
(define-method (fname (arg1 arg1-type) arg2))
      \end{lstlisting}

        où l'on peut optionnellement spécifier le type (symbole
        correspondant au nom de la classe) de l'argument
        passé. Lorsqu'aucune indication de type n'est donnée, le type
        par défaut aux données Scheme (\texttt{any-type}) est
        attribué.

      \subsubsection{Relation d'ordre sur les classes}
        \label{class-order-rel}
        Les fonctions génériques de notre système implante un
        polymorphisme plus simple que celui disponible avec un système
        aussi puissant que CLOS. Entre autre, nos fonctions génériques
        ne permettent pas d'utiliser des appels à d'autres instances
        comme \texttt{call-next-method}.

        Ainsi, lors de l'appel, l'instance de la fonction générique la
        \og meilleure \fg sera utilisée. Afin de trouver quelle
        instance est la meilleure pour des arguments de types donnés,
        une heuristique très simple est utilisée: ce sera l'instance
        qui devra avoir des paramètres tous super-types des paramètres
        actuels et dont la somme pour chaque argument formels des
        nombres de super classes sera la plus grande. Cette
        heuristique est utilisée pour sa simplicité et parce qu'elle
        semble donner un comportement prédictible, mais c'est le
        résultat d'un choix assez arbitraire.

      \subsubsection{Invoquation d'instances}

        Comme mentionné, la meilleure instance de la fonction
        générique invoquée, selon des arguments donnés, sera appellée
        lors d'une invoquation. Bien sûr, les fonction génériques
        effectuent du \og dispatch \fg multiple, et donc elles ne sont
        pas limitées à un seul argument (figure \ref{ex-multdisp}).

        \begin{figure}[h!]
          \begin{lstlisting}
> (define-generic (f (x A) y (z A) zz))
> (define-method (f (x A) y (z A) zz)
    (+ (A-a x) y (A-a z) (string->number zz)))
> (define-method (f (x A) y z zz)
    (+ (A-a x) y z zz))
> (f (make-A 1) 2 (make-A 3) "4")
10
> (f (make-A 1) 2 3 4)
10
          \end{lstlisting}
          \caption{Exemple d'invoquation d'une fonction générique et
            du \og dispatch \fg multiple en résultant.}
          \label{ex-multdisp}
        \end{figure}
        
        Par contre, il est aussi possible d'invoquer une instance
        particulière d'une fonction générique en utilisant un \og cast
        \fg. Dans l'exemple illustré dans la figure \ref{ex-cast},

        \begin{figure}[h!]
          \begin{lstlisting}
> (define-class A () (slot: a))
> (define-class B (A) (slot: b))
> (define-class C (B) (slot: c))
> (define-generic (f x))
> (define-method (f (x A)) (A-a x))
> (define-method (f (x B)) (B-b x))
> (f (make-C 1 2 3))
2
> (f (cast (make-C 1 2 3) 'A))
1
          \end{lstlisting}
          \caption{Exemple illustrant l'utilisation du \og cast \fg
            disponible appeler une instance spécifique d'une fonction
            générique.}
          \label{ex-cast}
        \end{figure}

      \subsubsection{Modes d'utilisations}

        Deux modes de programmations sont définis pour le système
        objet, tout particulièrement pour l'utilisation des fonctions
        génériques. Le premier mode, le mode itératif (par défaut),
        permet de définir et d'invoquer directement les nouvelles
        instances, comme illustré dans les figure \ref{ex-cast} et
        figure \ref{ex-multdisp}. Ce mode est très utile pour du
        développement interactif, comme c'est souvent le cas en
        Scheme. 

        Un second mode est aussi disponible, le mode manuel, qui
        permettrait d'installer les instances de fonctions génériques,
        sur demande, et non pas après chaque définitions de nouvelles
        instances. Ce mode devient inévitable lorsqu'on compile du
        code utilisant le système.

        On peut donc choisir le mode désiré en utilisant les formes
        spéciales:

        \begin{lstlisting}
(set-iterative-method-developpement!) ; par defaut
(set-manual-method-developpement!)
        \end{lstlisting}

        Il est fortement de recommandé de choisir le mode
        immédiatement après l'inclusion du système objet.

    \subsection{Outils additionnels disponibles à l'exécution}

      Plusieurs fonctionnalités sont disponible durant l'exécution du
      système d'objets afin de rendre plus agréable son
      utilisation. Les fonctions principales sont:

      \begin{itemize}
      \item[\texttt{(find-class 'class-name)}:] Permet d'obtenir le
        descripteur de classe de la classe nommée
        \texttt{class-name}, si elle existe.
      \item[]
      \item[\texttt{(class-desc-id class-descriptor)}:] Retourne le
        nom de la classe passée en paramètre.
      \item[]
      \item[\texttt{(class-desc-supers class-descriptor)}:] Retourne
        la liste des super classes directes de la classe donnée.
      \item[]
      \item[\texttt{(object? obj)}:] Prédicat vérifiant si l'objet
        donnée est une instance d'une classe du système.
      \item[]
      \item[\texttt{(get-class-id obj)}:] Retourne le nom symbolique
        de la classe de l'objet donnée. Si l'objet n'est pas une
        instance, alors \texttt{'any-type} est retourné.
      \item[]
      \item[\texttt{(instance-of? obj 'class-name)}:] Vérifie de
        manière efficace si l'objet est une instance directe de la
        classe associée au symbole donné.
      \item[]
      \item[\texttt{(object-light-copy obj)}:] Créé une copie légère
        de l'instance passée en paramètre. Une copie légère créé une
        nouvelle instance, ne fais que copier les pointeurs contenu
        dans l'instance.
      \end{itemize}

      Aussi, une forme spéciale (\texttt{update!}) est disponible pour
      faire une mise-à-jour incrémentale d'un champs d'instance. Une
      mise-à-jour incrémentale survient lorsque qu'on modifie un
      champs en fonction de sa valeur courante. La figure
      \ref{ex-update-a} illustre un exemple typique de mise-à-jour
      incrémentale et la figure \ref{ex-update-b} illustre un exemple
      d'utilisation de la forme spéciale équivalent.

      \begin{figure}[h!]
          \begin{lstlisting}
(define-class position () (slot: x) (slot: y))
(let ((current-position (make-position 3 4)))
    (position-x-set! current-position
                     (+ (position-x current-position) 5)))
          \end{lstlisting}
          \caption{Exemple typique de mise-à-jour incrémentale}
          \label{ex-update-a}
      \end{figure}

      \begin{figure}[h!]
          \begin{lstlisting}
(define-class position () (slot: x) (slot: y))
(let ((current-position (make-position 3 4)))
    (update! current-position position x (lambda (x) (+ x 5))))
          \end{lstlisting}
          \caption{Code équivalent utilisant la forme spéciale
            \texttt{update!}}
          \label{ex-update-b}
      \end{figure}

    Bien sûr, l'utilisation de cette macro n'est pas nécessaire, mais
    permet souvent de faire de manière plus courte une mise-à-jour
    incrémentale.

%\clearpage



%%%%%%%%%%%%%%%%%%%%%%%%%%%%%%%%%%%%%%%%%%%%%%%%%%%%%%%%%%%%%%%%%%%%%%%%%%%%%%%
\section{Implantation}

  Dans cette section, l'implantation du système objets décrit dans la
  section \ref{api} sera décrite. Ainsi, les structures de données et
  algorithmes seront expliqués afin d'expliquer le fonctionnement du
  système et les choix qui ont dû être fait afin de respecter le
  mieux possible les contraintes imposées initialement.

  \subsection{Classes et Objets}j
    \label{classes&objets}
    La forme spéciale de déclaration de classe (\texttt{define-class})
    est implanté dans notre système comme une macro Scheme. Cette
    macro prends les paramètre passés en entrés, soit le nom de la
    classe, la liste des super classes et les champs de celle-ci et
    effectue le traitement nécessaire afin de pouvoir générer le code
    permettant d'obtenir le comportement décrit plus haut.

    Le traitement nécessaire est relativement simple et se fait en 3
    étapes:

    \begin{enumerate}
    \item Traitement des champs hérités des super classes
    \item Traitement des champs directs de la classe
    \item Génération de code
    \end{enumerate}


                    %%%%%%%%%%%%%%%%%%%%%%%%%%%%%%
    \subsubsection{Traitement des champs hérités des super classes}
   
      Afin d'obtenir les champs hérités des super classes, les super
      classes sont passées dans l'ordre spécifié par
      l'utilisateur. Les champs de chaque super classes sont ajoutés à
      une \og eq? table \fg utilisé temporairement
      (\texttt{temp-field-table}). On vérifie aussi que les champs ne
      sont pas héritées plus d'une fois, sinon on retourne une
      erreur. Bien sûr, une vérification assure que toutes les classes
      sont des classes valides du système. La figure
      \ref{heritage-code} illustre ce traitement.
   
      \begin{figure}[h!]
        \begin{lstlisting}
  (for-each
   (lambda (super)
     (let ((super-field-indices
            (cond
             ((table-ref class-table super #f) =>
              (lambda (desc) (class-info-fi desc)))
             (else (error "Inexistant super class")))))
       (for-each
        (lambda (field-index)
          (if (not (table-ref temp-field-table (car field-index) 
                               #f))
              (table-set! temp-field-table
                          (car field-index) (cdr field-index))
              (error "Field already defined")))
        super-field-indices)))
   supers)
        \end{lstlisting}
        \caption{Code effectuant le traitement des champs hérités}
        \label{heritage-code}
      \end{figure}
      
      Maintenant que les champs hérités sont en mains, il ne suffit que de
      traiter les champs directs de la classe.

    \subsubsection{Traitement des champs directs de la classe}
      Par la suite, les champs directs de la classe sont traités. Le
      traitement effectue le \og parsing \fg des champs en s'assurant
      qu'ils sont bien formés. Par la suite, une vérification assure
      que le champs n'existe pas dans les champs hérités, si c'est le
      cas, alors le champs hérité est utilisé.

      Encore une fois, l'information sur les champs est ajouté dans la
      \og eq? table \fg temporaire.

      \begin{figure}[h!]
        \begin{lstlisting}
(define (process-field! field)
  (cond
   ((and (list? field)
         (>= (length field) 2))
    (let ((slot-type (case (car field)
                         ((slot: class-slot:) (car field))
                         (else (error "Bad slot type!"))))
          (slot-name (cadr field))
          (slot-options (cddr field)))
      ;; If a field is already provided by a super class
      ;; then the super class's is used ...
      (if (not (table-ref temp-field-table slot-name #f))
          (table-set! temp-field-table
                      slot-name
                      (make-slot slot-type
                                 (next-desc-index)
                                 slot-options)))))
   (else "ill-formed slot declaration")))
(for-each process-field! fields)
        \end{lstlisting}
        \caption{Code effectuant le traitement des champs hérités}
        \label{heritage-code}
      \end{figure}

      Lorsque toute l'information sur les champs est en main, il est
      maintenant possible d'effectuer la génération de code implantant
      la création et manipulation d'instances de cette nouvelle classe.

    \subsubsection{Génération de code}
      En ce qui a attrait à la génération du code exécuté, le logique
      est très simple. Elle se résume bien par le code effectuant le
      travail (figure \ref{class-code-gen})

      \begin{figure}[h!]
        \begin{lstlisting}
  (let* ((field-indices (sort-field-indices
                             (table->list temp-field-table)))
          (class-desc (gen-descriptor field-indices)))
    (table-set! class-table name 
                 (make-class-info field-indices class-desc))
    `(begin ,@(gen-accessors field-indices)
             ,@(gen-setters field-indices)
             ,(gen-predicate)
             ,(gen-instantiator field-indices)
             ,(gen-printfun field-indices)
             (table-set! ,(rt-class-table-name)
                          ',name
                          ,(class-desc-name name))))
        \end{lstlisting}
        \caption{Génération de code par un appel à \texttt{define-class}}
        \label{class-code-gen}
      \end{figure}

      Ainsi, on constate que la table temporaire de champs est
      transformée en structure de donnée qui sera conservée tout au
      long de la phase d'expansion macro, via une table globale
      contenant toutes les classes déclarées
      (\texttt{class-table}). Par la suite, les fonctions nécessaires
      sont générées à partir de ces données de manière à ce que
      l'accès au champs ce fasse rapidement, mais sans utiliser trop
      d'espace mémoire non plus. Plus de détails sur la structure des
      instances et l'accès au champs sont donnés dans la section
      \ref{inst-struct}.

      Il est à noter que le code généré par \texttt{gen-instantiator}
      créé une variable globale à l'exécution qui possédera un
      pointeur direct vers le descripteur de classe. On peut ainsi
      accéder très facilement aux champs de classes. Par contre, un
      pointeur vers le descripteur de classe est aussi placé dans une
      table de hachage (similaire à celle présente lors de l'expansion
      macro) afin de permettre plus de flexibilité.

                    %%%%%%%%%%%%%%%%%%%%%%%%%%%%%%  
  \subsection{Fonctions Génériques}
    L'implantation utilisée pour les fonctions générique est très
    simple, ce qui, nous croyons, assure de bonnes performances. De
    manière globale, les déclarations ne font que stocker des
    informations disponible durant l'expansion macro. Ensuite, il
    faudra appeler la forme spéciale
    \texttt{setup-generic-functions!} qui s'occupera de tout mettre en
    place correctement. Nous croyons que c'est une approche très
    simple parce qu'ainsi, lorsque \texttt{setup-generic-functions!}
    est appelée, nous savons avoir toute l'information disponible
    afin de générer le polymorphisme. Par contre, sinon après chaque
    déclaration de nouvelle instance de fonction générique, il est
    possible que le polymorphisme trouvé précédemment ne soit plus
    valide.

    Ainsi, il en résulte le besoin de créer les 2 modes d'utilisation
    du système de fonctions générique suggérés dans la section portant
    sur l'api: le mode interactif et le mode manuel. Dans le mode
    interactif, le polymorphisme est re-généré après \emph{chaque}
    déclaration de nouvelle instances de fonction générique. Ceci est
    très problématique lorsqu'on compile le code puisque comme décrit
    plus bas, le génération du polymorphisme est très coûteuse non
    seulement en calcul mais aussi en taille de code. Et donc, il en
    résulte en une explosion de la taille des exécutable lorsque le
    programme est compilé en mode itératif. Par contre, avec le mode
    manuel, il est possible de ne générer qu'une seule fois le
    polymorphisme, lorsque le programmeur sait que toutes les
    instances ont été déclarées et obtenir une taille de code
    acceptable.

    Nous allons maintenant donner plus de détails sur notre
    implantation des fonctions génériques.

    \subsubsection{Déclaration de fonctions génériques}
      La déclaration est extrêmement simple. Lorsque la macro
      \texttt{define-generic} est appelée, l'information sur le noms
      des paramètres et le nom de la fonction sont conservés dans une
      table de fonctions générique globale (\texttt{meth-table})
      existant durant l'expansion macro. La figure \ref{defgen}
      illustre le travail effectué lors de la déclaration d'une
      fonction générique.

      \begin{figure}[h!]
        \begin{lstlisting}
(define-macro (define-generic signature)
  (define name (meth-name signature))
  (define (parse-arg arg) ...)
  (define (args) (map parse-arg (cdr signature)))
  (table-set! meth-table name 
               (make-generic-function name (args)))
  ''ok)
        \end{lstlisting}
        \caption{Implantation de la déclaration de fonctions
          génériques}
        \label{defgen}
      \end{figure}
    
    \subsubsection{Déclaration d'instances de fonctions génériques}
      Le travail effectué pour la déclaration d'instance est aussi for
      simple. Il ne suffit qu'a analyser la signature de l'instance
      donnée afin d'en extraire le type de ses paramètre, et d'ajouter
      une représentation abstraite de la méthode dans la structure
      associée à la fonction générique existant durant l'expansion de
      macro. La figure \ref{defmeth} illustre grossièrement le travail
      effectué.

      \begin{figure}[h!]
        \begin{lstlisting}
(define-macro (define-method signature bod . bods)
  (define (parse-arg arg) ...)
  (define (parse-args args) (map-values parse-arg args))
  (with-exception-catcher
   (lambda (e) ...)
   (lambda ()
     (cond
      ((table-ref meth-table (name) #f) =>
       (lambda (gen-fun)
         (receive (args types) (parse-args (cdr signature))
           (generic-function-instances-add!
             gen-fun
             (make-method (name) types
                `(lambda ,args ,bod ,@bods)))
           (case mode
                  ((iterative) '(setup-generic-functions!))
                  ((manual)    ''ok)
                  (else        (error ...))))))
      (else (raise unknown-meth-error))))))
        \end{lstlisting}
        \caption{Implantation des déclarations d'instances de
          fonctions génériques}
        \label{defmeth}
      \end{figure}

    \subsubsection{Génération du polymorphisme}
      Finalement, lorsque le moment est opportun, il faut mettre en
      place les instances de fonctions génériques accumulées durant
      l'expansion macro. La génération de code ce distingue en 3 étapes:

      \begin{enumerate}
      \item Mise en place des structures qui accueilleront les
        fonctions génériques.
      \item Insertion des instances de fonctions génériques déclarées
        dans l'endroit correspondant aux types de cet instance.
      \item En utilisant le produit cartésien de tous les types
        (classes), insérer les instances les plus appropriées et ce
        pour chaque fonctions générique.
      \end{enumerate}

      Pour la première étape, les structures de données décrite dans
      la section \ref{genfun-struct} seront initialisées. De plus, des
      méta-fonctions génériques pour chaque déclarations seront
      installées. Le but de celles-ci est de faire le \og dispatch \fg
      durant l'exécution vers la bonne instance de la fonction
      générique. Du code correspondant à l'expansion de la
      méta-fonction pour une fonction générique \texttt{f} est
      illustré dans la figure \ref{meta-genfun}.
      
      \begin{figure}[h!]
        \begin{lstlisting}
> (##decompile f)
(lambda (x)
  (let ((types (map get-class-id (list x))))
    (cond ((table-ref f-meth-table types #f)
            => (lambda (method) 
                 (apply method (map uncast (list x)))))
           (else (error "Unknown method")))))
        \end{lstlisting}
        \caption{Expansion de la méta-fonction pour une fonction
          générique \texttt{f}}
        \label{meta-genfun}
      \end{figure}
      
      Pour la deuxième étape, chacune des instances déclarées seront
      directement insérées pour les types spécifiés, dans la structure
      de la fonction générique. Par exemple, la figure
      \ref{ex-insertion-directe-meth} illustre une insertion directe
      d'une instance déclarée de la fonction générique \texttt{f}
      ayant comme paramètre la variable \texttt{x} de type \texttt{D}.

      \begin{figure}[h!]
        \begin{lstlisting}
(table-set! f-meth-table '(D) (lambda (x) (D-inst x)))
        \end{lstlisting}
        \caption{Insertion directe d'une instance de la fonction
          générique \texttt{f} pour un paramètre \texttt{x} de type
          \texttt{D}.}
        \label{ex-insertion-directe-meth}
      \end{figure}

      Finalement, il ne reste plus qu'a \og remplir les trous \fg,
      \textit{i.e.} de générer le polymorphisme en ajoutant pour
      toutes les combinaisons de types de paramètres pour une fonction
      générique donnée l'instance la meilleure (la plus spécialisée).

      Pour ce faire, nous implantons l'algorithme \ref{algo-poly}. Cet
      algorithme possède beaucoup de \og magie noire \fg afin de faire
      fonctionner le polymorphisme, mais le résultat est concluant. En
      résumé, on va placer l'instance la plus appropriée pour un
      tuples de types appartenant au produit cartésiens de toutes les
      possibilités de sous-types existante pour cette instance. Afin de
      faire le triage, ou l'ordonnancement pour trouver l'instance la
      plus spécifique, on utilise la même idée que celle décrite dans
      la section \ref{class-order-rel} traitant sur l'ordre relatif
      entre les classes. La liste triée de l'instance la plus
      spécifique à la moins spécifique est renversée de manière à ce
      que l'insertion la plus spécifique soit en \emph{dernier} et
      donc soit effective. Il en résulte donc que non seulement
      l'algorithme est terriblement inneficace, mais aussi qu'il
      génère beaucoup de code inutile.

      Par contre, nous croyons que ce soit acceptable dans la mesure
      où la génération du polymorphisme n'est pas appelée
      souvent. Dans le mode manuel, elle devrait être invoquée une
      seule fois. Dans le mode interactif, elle sera invoquée après
      chaque déclaration d'instance, mais puisque l'utilisateur sera
      typiquement dans une \og REPL \fg, le calcul et code généré
      superflu ne sera pas visible. 

      \begin{algorithm}
        \caption{Algorithme de génération de polymorphisme}
        \label{algo-poly}
        \begin{algorithmic}
        \FORALL{$g \in$ declared generic function which has instances}
          \FORALL{$m \in $ reverse sorted methods of $g$}
            \FORALL{$types \in$ possible valid argument types of $m$}
              \STATE (table-set! $g$-method-table $types$ $m$)
            \ENDFOR
          \ENDFOR
        \ENDFOR
        \end{algorithmic}
      \end{algorithm}
      

                    %%%%%%%%%%%%%%%%%%%%%%%%%%%%%%
  \subsection{Structures de données à l'exécution}
    Les structures de données utilisées pour implanter les classes,
    fonctions génériques et instances de ces deux derniers à
    l'exécution sont décrite dans cette section.
  
    \subsubsection{Descripteur de classes}
      Les descripteurs de classes sont les objets représentant la méta
      entité qu'est une classe. Le rôle que joue le descripteur dans
      notre système est primordial, car c'est lui qui permet de dire
      au fonctions accédant aux champs où retrouver ces derniers. On
      peut résumé la structure d'un descripteur de classe par le
      vecteur Scheme suivant:

      \begin{lstlisting}
#(nom-classe liste-supers . indexes)
      \end{lstlisting}

      Nous avons consciemment utilisé le point comme un abus de
      notation dans ce vecteur afin de représenter que le reste du
      vecteur correspond a des indices pour \emph{tout les champs de
        toutes les classes}. Lorsque les champs sont ajoutés à un
      descripteur de nouvelle classe, les nouveaux champs sont ajoutés
      à la suite des champs précédemment existant dans toutes les
      classes déjà déclarées. Ainsi, chaque champs possède un index
      \emph{unique} dans les descripteurs de classes.

      Ce choix résulte forcément en de larges descripteurs de classes,
      surtout lorsqu'il y a beaucoup de classes ou des classes
      possédant beaucoup de champs. Par contre, ceci permet d'avoir
      toute la flexibilité désirée pour la création d'instances. 

      Donc, dans le descripteur de classe, à l'index prédéterminer pour
      un champs donné, on y retrouvera l'index où se trouve ce champs en
      question dans une instance de la classe. On pourra donc avoir des
      instances de taille constante. Le coût de se choix résulte par
      contre, en une indirection supplémentaire pour l'accès à un champs
      (il faut aller chercher où se trouve le champs dans l'instance à
      partir de l'information contenue dans le descripteur de classe).

      Afin d'illustrer ce qui vient d'être énoncé, l'exemple présenté
      dans la figure \ref{ex-cdesc} illustre les descripteurs de classes
      créés lorsque le système est \og frais \fg et qu'aucune autre
      classe n'existe au préalable. On constate que par exemple, pour la
      classe \texttt{C}, l'index d'instance du champs \texttt{zz} se
      retrouve après trois \og trous \fg correspondant aux champs des
      classes \texttt{A} et \texttt{B} et sa valeur d'index d'instance
      est de 1.

      \begin{figure}[h!]
        \begin{lstlisting}
> (include "class.scm")
object-system-loaded
> (define-class A () (slot: a))
> (find-class? 'A)
#(A () 1)
> (define-class B (A) (slot: b) (slot: c))
> (find-class? 'B)
#(B (A) 1 2 3)
> (define-class C () (slot: zz))
> (find-class? 'C)
#(C () unknown-slot unknown-slot unknown-slot 1)
        \end{lstlisting}
        \caption{Exemple illustrant le comportement des descripteurs de
          classes. Les descripteurs de classes grossissent de plus en
          plus qu'il y a de classes déclarées.}
        \label{ex-cdesc}
      \end{figure}

    \subsubsection{Structure des instances}
      \label{inst-struct}
      Les instances sont représentées de manière très simple. Il
      s'agit d'un vecteur dont le premier champs est un pointeur vers
      le descripteur de classe et les autres contiennent les champs
      d'instances, dans l'ordre tel que décrit dans l'api (voir
      section \ref{api}) ou dans le traitement déclarations de classes
      (voir section \ref{classes&objets}). En reprenant les
      déclarations de classes de la figure \ref{ex-cdesc}, la figure
      \ref{ex-struct-inst} illustre la structure des instances de ces
      classes \texttt{A}, \texttt{B} et \texttt{C}.

      \begin{figure}[h!]
        \begin{lstlisting}
> (make-A 'allo)
#(#(A () 1) allo)
> (make-B 'allo 'toto 'titi)
#(#(B (A) 1 2 3) allo toto titi)
> (make-C 'bonjour)
#(#(C () unknown-slot unknown-slot unknown-slot 1) bonjour)
        \end{lstlisting}
        \caption{Illustration de la structure des instances de classes à
          partir des définitions de classes données dans la figure
          \ref{ex-cdesc}.}
        \label{ex-struct-inst}
      \end{figure}

      Malgré le fait que les instances semblent d'augmenter de tailles
      en fonction du nombres de classes précédemment définies, ce n'est
      pas le cas puisqu'il ne s'agit que d'un pointeur vers le
      descripteur de classe.

      Ainsi, un accès à un champs se fait en temps constant ($\in
      O(1)$). Il faut par contre suivre deux indirections: une pour
      obtenir le descripteur de classe et une seconde par le
      descripteur de classe afin de savoir où se trouve le champs dans
      l'instance. Par contre, l'accès aux champs de classes sont
      directs puisque ceux-ci sont stockés directement dans le
      descripteur de classe (où il y aurait normalement l'index
      d'instance). Ce concept est illustré dans l'exemple
      \ref{ex-slot-access}, où le code effectuant l'accès à des champs
      d'instance et de classe est donnée pour une classe exemple.

      \begin{figure}[h!]
        \begin{lstlisting}
> (define-class D () (slot: inst) (class-slot: clas))
> (find-class? 'D)
#(D () 1 unbound-class-slot)
> (##decompile D-inst)
(lambda (#:obj1373) (vector-ref #:obj1373 
                        (vector-ref (vector-ref #:obj1373 0) 2)))
> (##decompile D-clas)
(lambda () (vector-ref D-class-descriptor 3))
> (D-clas-set! 'allo)
> (find-class? 'D)
#(D () 1 allo)
        \end{lstlisting}
        \caption{Illustration du code effectuant l'accès aux champs
          d'instances et de classes.}
        \label{ex-slot-access}
      \end{figure}

    \subsubsection{Structure des fonction génériques et de leurs instances}
      \label{genfun-struct}
      Pour chacune des fonctions génériques déclarées, une variable
      globale correspondant à cette fonction générique pointera vers
      une table de hachage. Cette table sera remplie par chacune des
      instances de cette fonction générique. Une liste des types de
      chaque argument d'une instance sera utilisée comme clé et la
      valeur correspondra à la fermeture associée.

      Ainsi, lorsque la fonction générique sera appelée via une
      méta-fonction, une recherche dans la table sera effectuée et si
      une instance est trouvée selon le type des arguments actuels
      utilisés, alors cette fermeture sera appelée avec ces arguments
    

%\clearpage


 
%%%%%%%%%%%%%%%%%%%%%%%%%%%%%%%%%%%%%%%%%%%%%%%%%%%%%%%%%%%%%%%%%%%%%%%%%%%%%%%
\section{Limitations et Travail à Compléter}
  L'implantation du système orienté objet présenté comporte plusieurs
  limitations quand à son utilisation. La principale limitation serait
  certainement reliée au fait que le système ne fonctionne pas de
  manière inter-module. Donc il n'est pas possible d'utiliser des
  classes déclarées dans un autre fichier. Nous croyons toutefois
  qu'il serait difficile d'ajouter une telle fonctionnalité à notre
  système sans ajouté un coût considérable pour l'accès aux champs
  d'instances.

  Aussi, le système est peu adapté à des changements dans son
  fonctionnement. En effet, afin d'ajouter des nouvelles
  fonctionnalités au système, il faut plonger dans et modifier une
  grande partie du code. Ainsi, l'ajout de protocole de méta-objets
  tel que décrit dans \textit{The Art Of The MetaObject
    Protocol}~\cite{AOMOP} serait un grand atout pour le système. Non
  seulement, l'introspection serait beaucoup plus aisée, mais aussi la
  modification du système pourrait se faire de manière très modulaire.

  De plus, aucune comparaison de performances n'a pu être effectuée
  sur le système objet. Ainsi, nous croyons que les performances
  durant l'exécution, \textit{i.e.} coût de création d'instances
  d'objets, d'accès aux champs ou des appels de fonctions génériques,
  sont acceptables, mais une comparaison de \og benchmark \fg nous
  permettrait d'en être encore plus certains. De plus, cela nous
  donnerait la chance de pouvoir cibler les points plus faible du
  système afin d'être mieux conscient des goulots d'étranglements.

  Finalement, notre implantation utilise de manière abondante les
  tables de hachages de Gambit-C lors de l'expansion des macros. Il en
  résulte donc que notre système ne pourra pas être portable sur
  d'autres implantations de Scheme. Par contre, si nous avions notre
  propres librairies de tables de hachages, et que cette librairie
  soit portable, nous croyons que le système serait facile à porter
  sur toute implantation de Scheme supportant les macro à la
  \texttt{define-macro}.

%\clearpage

%%%%%%%%%%%%%%%%%%%%%%%%%%%%%%%%%%%%%%%%%%%%%%%%%%%%%%%%%%%%%%%%%%%%%%%%%%%%%%%
\section{Conclusion}
  En résumé, nous avons utiliser le système de macro de Scheme afin de
  créer un système de programmation orienté objet, au dessus de
  Scheme. Ainsi, il en résulte que le système est très modulaire, en
  le sens, qu'il peut être utilisé à l'aide d'une simple inclusion de
  code source (\texttt{(include "class.scm")}). 

  Pour ce faire, nous avons utilisé des vecteurs comme structures de
  donnée représentant les objets et les classes et des tables de
  hachages pour contenir les instances de fonctions génériques. Grâce
  à la structure choisie, l'accès aux champs d'instances de classes se
  fait en trois références de vecteurs et l'appel de fonction
  générique se fait en une référence à une table de hachage de test
  \texttt{equal?}.

  Grâce à l'aide d'un algorithme très naïf, nous avons réussi à faire
  un fonctionner le polymorphisme dans notre système de classe à
  héritage multiple. Malgré une complexité élevée et une génération de
  code inutile, nous croyons que le système est utilisable, mais
  pourrait bénéficier d'une amélioration de l'algorithme de génération
  du polymorphisme.

  Ainsi, notre système d'objet possède beaucoup de fonctionnalités
  provenant de CLOS, entre autre l'héritage multiple des classes et
  des fonctions génériques polymorphes à \og dispatch \fg
  multiple. Donc tout usagers ayant des connaissances préalables à
  CLOS ne devrait pas avoir de difficulté à utiliser le système.

  Finalement, notre système dépend directement de l'implantation de
  Gambit-C, mais faiblement et pourrait donc sans trop de difficulté
  être porté à toute implantation de Scheme supportant les macros à la
  \texttt{define-macro} tel que disponible dans Gambit-C.

  Nous croyons que l'implantation de ce système objets démontre bien
  la puissance et l'expressivité du langage Scheme. En effet, dans un
  délais relativement court (deux à trois semaines), un système objet
  relativement complet, fonctionnel et performant a pu être créé selon
  des besoins spécifiques. Ainsi, malgré le fait que le langage
  Scheme est très léger, il peut facilement être adapté, même pour des
  besoins qui peuvent sembler complexe, voir inaccessibles.

  Il serait très intéressant pour les auteurs de pouvoir maintenant
  ré-écrire tout le système objets, de manière à pouvoir ajouter un
  protocole de méta-objets et améliorer l'algorithme de génération du
  polymorphisme. Une réécriture complète du système serait
  probablement plus enrichissante qu'une modification puisque
  l'expérience acquise lors de l'écriture de ce système apporterait
  beaucoup pour le design lors d'une réécriture. Il en résulterait
  alors un système très intéressant qui pourrait servir à de nombreux
  programmeurs Scheme.


%% \clearpage

%%%%%%%%%%%%%%%%%%%%%%%%%%%%%%%%%%%%%%%%%%%%%%%%%%%%%%%%%%%%%%%%%%%%%%%%%%%%%%%
\chapter{Conclusion}
  
  En résumé, nous avons étendu l'implantation du système Scheme
  Gambit-C en lui ajoutant plusieurs améliorations. 

  Dans un premier temps, une analyse de flot de contrôle de type
  \textit{0-cfa} a été ajoutée aux analyses et transformations de code
  déjà existante. Cette analyse permet d'obtenir une approximation du
  flot de contrôle du programme, de plus qu'une approximation du flot
  des données résultant de ce flot de contrôle. Les approximations
  obtenues sont conservatrices, afin de bien saisir la sémantique du
  programme analysé.

  Ensuite, cette analyse a été utilisé dans l'ajout d'optimisations
  d'évaluations partielles reposant sur le flot des données obtenu
  précédemment. --- TODO Jeremie ---

  Finalement, un système orienté objet polymorphe à héritage multiple,
  à la CLOS, a été écrit en utilisant les macros Scheme. Ce système
  fut conçu spécialement pour avoir une accès rapide aux champs des
  objets, tout en faisant en sorte que les objets soient de taile
  minimale. 

  Bien sûr, beaucoup d'aspects de notre projet méritent d'être
  re-travaillés ou améliorer. Entre autres, plusieurs nouvelles
  évaluations partielles pourraient être ajoutées. Aussi, une
  ré-écriture du système objet lui permettant de fournir un protocole
  de méta-objets donnerait beaucoup plus de flexibilité à celui-ci.

  Malgré tout, nous croyons que notre projet constitue une
  contribution permettant d'améliorer le système Gambit-C, tant au
  point de vue d'une compilation plus optimisée qu'au point vue plus
  complétude.

%% \clearpage

%%%%%%%%%%%%%%%%%%%%%%%%%%%%%%%%%%%%%%%%%%%%%%%%%%%%%%%%%%%%%%%%%%%%%%%%%%%%%%%

\addcontentsline{toc}{section}{R\'ef\'erences}
\bibliographystyle{abbrv}
\bibliography{bibliography}



\end{document}
